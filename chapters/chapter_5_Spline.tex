\chapter{Splines}
\section{Definition and Simple properties}
\paragraph{样条定义}令$T^*:=(t_i^*)_1^s$或$T^*:=(t_i^*)_{=\infty}^\infty$是$\mathbb{R}$上严格递增的有限序列或双无穷序列,在第二种情况下,假定$|t_i^*|\to\infty,i\to\infty$.若在每个区间$(t_i^*,t_{i+1}^*)$上,函数$S$是一个次数$\leq m=r-1$的多项式,且至少有一个区间上次数恰为$m$,则称$S$是$r$阶且断点为$T^*$的样条.

\paragraph{样条在断点处光滑度}样条在断点$t_i^*$处的光滑度$m_i$定义如下:
\begin{itemize}
    \item 若$S$在$t_i^*$处不连续,则令$m_i=0$;
    \item 否则,$m_i$是满足$0<m_i\leq r$的最大整数,使得$S$在$t_i^*$的某个邻域内属于$C^{(m_i-1)}$,也就是说$S$的各阶导数直到$m_i-1$阶都连续.
\end{itemize}


\paragraph{样条空间}给定区间$A = [a,b]$或$A = \mathbb{R}$,以及断点序列$T^*$,可以在$A$上构造样条空间.记:$S_r^*(A)$
表示$A$上所有阶数$\leq r$的样条构成的空间;$S_r^*(T^*,A)$表示$A$上所有阶数$\leq r$且断点都包含在$T^*$中都样条构成的空间.

\paragraph{Schoenberg空间}
给定断点集$T^*$和一系列$m_i(0\leq m_i<r)$,我们定义$A$上的Schoenberg空间:他由所有阶数$\leq r$、断点包含在$T^*$中,并且$t_i^*$处的光滑度至少为$m_i$的样条$S$组成. 不过,通常不是直接使用$m_i$,而是使用所谓的亏格(defect)
\begin{equation*}
    k_i:= r-m_i,
\end{equation*}
因为$k_i$更好的描述了$S$在$t_i^*$处自由度的个数.这个Schoenberg空间记作
\begin{equation*}
    S_r:=S_r(R^*,\boldsymbol{k},A),\quad \boldsymbol{k}:=(k_i).
\end{equation*}

Schoenberg空间有如下性质
\begin{enumerate}
    \item $S_r(T^*,\boldsymbol{1},A)$是任意其他 Schoenberg空间$S_r(T^*,\boldsymbol{k},A)$的子空间;
    \item 若$S\in S_r(T^*,\boldsymbol{k},A)$,则其任意原函数$S_1\in S_{r+1}(T^*,\boldsymbol{k},A)$,其中亏格$\boldsymbol{k}$相同;
    \item $S_r(T^*,\boldsymbol{k},A)$总包含所有次数$\leq r-1$的多项式$P_{r-1}$,另一方面,他又是$S_r(T^*,\boldsymbol{r},A)$的子空间(即没有任何光滑性要求,只是分段多项式).
\end{enumerate}

\paragraph{样条空间的基}
在情形 $A=[a,b]$ 下,可以借助截断幂构造 $Y := S_r(T^*,\mathbf{k},A)$ 的一个自然基. 通常可以按以下方式找到一组基. 设$dim(Y)= N$,我们在 $Y$ 中找到元素
$S_1,\ldots,S_N$ 以及线性泛函 $a_1,\ldots,a_N$,满足
\begin{enumerate}
    \item \[
  a_j(S_i)=0,\quad i\neq j;\qquad a_j(S_j)=1,
\]
\item 
\[
  a_j(S)=0,\quad j=1,\ldots,N\Rightarrow S=0,
\]
\end{enumerate}
在这种情形下有
\begin{equation}\label{eq:dual-expansion}
  S = \sum_{j=1}^{N} a_j(S)\,S_j ,
\end{equation}
其中 $a_j$ 称为 $S_j$ 的\emph{对偶泛函}. 


\begin{theorem}[样条空间的基]\label{thm:样条空间的基}
若区间 $A=[a,b]$ 有限,则空间 $S_r(T^*,\mathbf{k},I)$ 具有如下基
\begin{align}
S_{-j}(x) &:= \frac{(x-a)^j}{j!}, && j = 0,\ldots,r-1, \label{eq:1.3a}\\
S_{i,j}(x) &:= \frac{(x-t_i^*)_+^{\,j}}{j!}, && j = r-k_i,\ldots,r-1,\quad i = 1,\ldots,s, \label{eq:1.3b}
\end{align}
其对应的对偶泛函为
\begin{align}
a_{-j}(S) &:= S^{(j)}(a), && j = 0,\ldots,r-1, \label{eq:1.4a}\\
a_{i,j}(S) &:= S^{(j)}(t_i^{*+}) - S^{(j)}(t_i^{*-}), && j = r-k_i,\ldots,r-1,\quad i = 1,\ldots,s. \label{eq:1.4b}
\end{align}
特别地,有
\begin{equation}
\dim S_r(T^*,\mathbf{k},A) = n + r,\qquad 
n := \sum_{i=1}^r k_i. \tag{1.5}
\end{equation}
\end{theorem}

\begin{proof}
令 $S\in S_r(T^*,\mathbf{k},I)$. 若对所有 $i=1,\ldots,S$ 以及 
$j = r-k_i,\ldots,r-1$ 都有 $a_{i,j}(S)=0$,则
\[
S,\ldots,S^{(r-1)}
\]
在每个 $t_i^*$ 处连续,因此 $S^{(r-1)}$ 为常数. 于是 $S^{(r)}\equiv 0$,
从而 $S$ 在 $A$ 上必为次数 $< r$ 的多项式. 若再对 $j=0,\ldots,r-1$ 都有
$a_{-j}(S)=0$,则该多项式所有阶导数在 $a$ 处均为零,只能是零函数,
即 $S\equiv 0$. 其余部分显然成立. 
\end{proof}


注:$k_i$表示每个节点处的自由度,对于多项式空间来说,基的个数谁$r$,样条空间每个节点$k_i$个自由度,也就是$k_i$个基.



根据定理\ref{thm:样条空间的基},$S\in S_r$,可以被写成
\begin{equation*}
    S(x) = P_{r-1}(x)+\sum_{i=1}^s\sum_{j=1}^{k_i}c_{i,j}(x-t_i^*)_+^{r-j}
\end{equation*}



\paragraph{样条基本不等式}
样条满足一些与多项式类似的基本不等式. 这里我们只给出 Nikol’skii 与 Markov 不等式在 Schoenberg 空间
$$
S_r(T^*,\mathbf{k},A),\qquad A=[a,b],
$$
中的变形,其中断点集合为
$$
T^* := \{t_j^*\}_{j=1}^s.
$$
我们约定
$$
t_0^* := a,\qquad t_{s+1}^* := b.
$$

\begin{theorem}
若断点 \(T^*\) 满足
\begin{equation}\label{eq:1.7}
\delta_0 \le \lvert t_{j+1}^* - t_j^* \rvert \le \delta,\qquad  j = 0,\dots,s,
\end{equation}
则对任意
$$
S\in S_r(T^*,\mathbf{k},A),\qquad A=[a,b],
$$
有
\begin{equation}\label{eq:1.8}
\lVert S\rVert_p \le C\,\delta^{1/q-1/p}\,\lVert S\rVert_q,\qquad
p_0 \le q \le p \le \infty,
\end{equation}
其中常数 \(C = C(p_0,r)\);并且当 \(k=1,\dots,r-1\) 时,还有
\begin{equation}\label{eq:1.9}
\lVert S^{(k)}\rVert_p \le C\,\delta_0^{-k}\,\lVert S\rVert_p,\qquad 0<p\le\infty,
\end{equation}
这里的常数 \(C = C(r)\). 
\end{theorem}

\section{B-样条}
定理\ref{thm:样条空间的基}中,把样条表示成若干截断幂级数之和,从一般观点看其实并不好用,因为截断幂级数的支撑集很大.因此需要引入另一组基,这就是B-样条(Basic splines).

\paragraph{B-样条定义}我们记$[x_0,\dots,x_n]f$表示$f$在这些点的$n$阶差商,我们定义B-样条为
\begin{equation*}
    M(x):=M(x;x_0,\dots,x_r):=r[x_0,\dots,x_r](\cdot -x)_+^{r-1}
\end{equation*}
其中的点$\{x_0,\dots,x_r\}$称为$M$的结点.
